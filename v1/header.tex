
\begin{abstract}
    We have previously presented what we called a {\em logical graphical model}.
    In this work, we further clarify that an important aspect of the model is that it is {\em probabilistic}.
    Thus, we will now begin to refer to this as a ``logical probablistic model''.
    The logical probabilistic model is a kind of a Bayesian Network.
    One important problem that has always been a blocker for the practical use of Bayesian Networks is that the {\em structure} of a Bayesian Network must somehow be {\em specified}.
    This is a problem that Neural Networks do not have.
\end{abstract}